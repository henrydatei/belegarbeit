\chapter{Allgemeine Angaben}
\label{chap:allgemeine_Angaben}

Die folgenden Angaben beziehen sich auf Vorgaben des Lehrstuhls für
Wirtschaftsinformatik | Business Intelligence Research der Technischen Universität Dresden.\\
Das folgende Kapitel beschäftigt sich dabei im Speziellen mit der Gliederung einer wissenschaftlichen Arbeit und den einzelnen Verzeichnissen.

\section{Gliederung}\index{Gliederung}
\label{chap:gliederung_Arbeit}

Eine wissenschaftliche Abschluss- oder Seminararbeit sollte wie folgt gegliedert sein:

\begin{itemize}
\item Einleitung
\item Kapitel 1
\item . . .
\item Kapitel n
\item Fazit (Zusammenfassung und Ausblick)
\item Literaturverzeichnis
\item Anhang
\end{itemize}

Die \textbf{Einleitung}\index{Einleitung} soll das zu behandelnde Thema vorstellen und in den übergeordneten
Sachzusammenhang einordnen. Sie soll den Leser zum Lesen der Arbeit motivieren und Ziele, Inhalte und Ergebnisse der
einzelnen Kapitel skizzenhaft vorstellen. 

Der \textbf{Hauptteil}\index{Hauptteil} der Arbeit ist in mehrere Kapitel zu
unterteilen, die dann ihrerseits wiederum verschiedene Abschnitte und Unterabschnitte (evtl. mit
Mehrfachuntergliederungen) aufweisen können. Die Kapiteleinteilung muss systemlogisch angelegt sein,
d.~h., kein Punkt oder Unterpunkt kann alleine (ohne einen weiteren korrespondierenden,
gleichgeordneten Punkt) stehen. Siehe dazu die Nummerische Gliederung nach dem Abstufungsprinzip (vgl.
\cite{Theisen2006}, S.~102~f.). Außerdem sollten die einzelnen Punkte (Kapitel) vom Umfang her gleichumfänglich sein,
um eine \glqq Klumpenbildung\grqq \ zu vermeiden. 

Im \textbf{Fazit}\index{Fazit} werden die Ergebnisse der Arbeit noch
einmal zusammengefasst, ggf. weiterführende oder angrenzende Problemstellungen aufgezeigt bzw. diskutiert. Weiterhin
kann auf nicht behandelte aber als relevant erachtete Problemstellungen hingewiesen werden.

\section{Verzeichnisse}\index{Verzeichnis}
\label{chap:verzeichnisse}
Werden in der Arbeit viele verschiedene Abkürzungen verwendet, so sollten diese -- auch im Falle von Zeitschriften --
in einem \textbf{Abkürzungsverzeichnis}\index{Verzeichnis!Abkürzungs-}\index{Abkürzungsverzeichnis} aufgelistet und
kurz erklärt werden. Hierbei werden Abkürzungen, welche im Duden zu finden sind (\glqq z.~B.\grqq , \glqq s.\grqq, etc.) nicht aufgelistet.
 Ebenso müssen ein
\textbf{Abbildungs}\index{Verzeichnis!Abbildungs-}\index{Abbildungsverzeichnis}-
und \textbf{Tabellenverzeichnis}\index{Tabellenverzeichnis}\index{Tabellenverzeichnis} (unter Angabe der jeweils
zugehörigen Seitenzahlen) vorhanden sein, wenn in die Arbeit Abbildungen und/oder Tabellen eingearbeitet sind.
In den meisten Fällen schließt die Arbeit mit dem
\textbf{Literaturverzeichnis}\index{Verzeichnis!Literatur-}\index{Literaturverzeichnis} ab. Bei Verwendung von vielen
aufeinander folgenden oder umfangreichen Tabellen und Abbildungen oder einer ausführlichen formelmäßigen Herleitung
(z.~B. zur Klärung oder Verdeutlichung einer bestimmten Aussage) sowie ergänzenden Materialien und Dokumenten in der
Arbeit sollten die betreffenden Teile in einem \textbf{Anhang}\index{Anhang} (im Anschluss an das
Literaturverzeichnis) aufgeführt werden, so dass dadurch der Lesefluss nicht gestört wird. Der Anhang kann (wenn dies
inhaltlich sinnvoll erscheint) in mehrere Teile gegliedert sein. Die einzelnen Kapitel und Abschnitte der Arbeit sind
unter Angabe der Seitenzahl in einem \textbf{Inhaltsverzeichnis}\index{Inhaltsverzeichnis}\index{Verzeichnis!Inhalts-}
aufzulisten, das der eigentlichen Arbeit vorangestellt wird.\\

Die gesamte Arbeit (einschließlich Einleitung und Literaturverzeichnis) muss mit einer
\textbf{Seitennummerierung}\index{Seitennummerierung}\index{Nummerierung!Seiten-} bzw. Paginierung\index{Paginierung}
versehen werden. Dabei ist nur der Textteil mit fortlaufenden \textbf{arabischen Ziffern}\index{Ziffern!arabisch} zu
nummerieren, und die restlichen Teile der Arbeit (Verzeichnisse und Anhang) sind wiederum fortlaufend mit
\textbf{römischen Ziffern}\index{Ziffern!römisch} zu nummerieren, wobei das Literaturverzeichnis dem Textteil zuzurechnen ist (vgl.
\cite{Theisen2006}, S.~179~f.).\\

\section{Exposé}
\label{chap:expose}
Vor Beginn des Schreibens der Arbeit ist mit dem jeweiligen Betreuer das Vorgehen für die Bearbeitung der Arbeit abzuklären. Dazu muss nach Ausgabe des Themas in der Regel binnen zwei Wochen ein Exposé erstellt werden, in dem dargestellt wird, welche Ziele mit welchen Forschungsmethoden verfolgt werden (Forschungsdesign). Zu Informationen zum Forschungsdesign und der allgemeinen Wissenschaftstheorie siehe Anhang \ref{chap:Wi_Th}.
Die Bestandteile des Exposés sind ausführlich und mit einem Beispiel im Anhang \ref{chap:a_expose} beschrieben. Das Exposé ist mit dem jeweiligen Betreuer zu besprechen und dient in seiner abgesegneten Form als Forschungsgrundlage für die anzufertigende Arbeit.



\section{Eidestattliche Erklärung}
\label{chap:diplomarbeiten}

Bei Diplom-, Studien- und Seminararbeiten muss eine \textbf{Eidesstattliche
Erklärung}\index{Erklärung!Eidesstattliche}\index{Eidesstattliche Erklärung} am Ende der Arbeit eingefügt werden. Der
genaue Wortlaut für Arbeiten mit WORD ist der Vorlage im Anhang \ref{eides} zu entnehmen. Bei der
\LaTeX-Vorlage wird die Eidesstattliche Erklärung bei Benutzung des entsprechenden Argumentes automatisch erzeugt. 